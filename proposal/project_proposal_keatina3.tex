\documentclass[12pt]{article}
\usepackage{geometry}
\usepackage{amsmath}
\usepackage{amssymb}
\usepackage{enumitem}
\usepackage{fancyhdr}
\pagestyle{fancy}
\setlength{\headheight}{20pt}

\renewcommand{\refname}{\normalsize References}

\lhead{Project Proposal}
\chead{Alex Keating}
\rhead{\today}

\begin{document}

\title{Project Proposal}

\subsection*{Improved Iterative Relaxation Method in FEM-SES:}
\subsubsection*{Supervisors:}
Kirk Soodhalter, Jose Refojo.
\subsubsection*{Overview:}
My proposed topic for my MSc. project is based around replicating and extending the work done in the paper \textit{Alternate Parallel Processing Approach for FEM}, a paper which develops a novel finite element method for GPUs, namely FEM-SES, or \textit{finite element method single element solution}~\cite{femses}. The overall approach of this paper can be looked at as removing the global stiffness matrix creation step of the traditional FEM, with the aim of improving parallelism, by developing an element-based solution. FEM-SES ahderes to the classical approach up to the point of creating a global stiffness matrix by appplying the boundary conditions, instead it applies boundary conditions to element matrices, and uses a Jacobi-relaxation scheme to solve these and uses a novel node-based weighting to achieve the global solution from the element solutions.\\
The paper manages to massively improve on the amount of parallelism traditionallly seen in partitioning, domain decomposition and multigrid techniques for FEM. It also hugely reduces the amount of data needed to be sent to the host as all that is needed at each iteration until convergence is a single value. However, from profiling done in the paper, the \textit{2-step iterative relaxation} is by a long shot the most dominant kernel in the algorithm, which is where I was potentially looking at extending the paper. FEM-SES uses a Jacobi relaxation, which is not the fastest to converge, and so I am proposing to investigate applying other more efficient/robust solvers, looking at Gauss-Seidel, SOR or possibly an entire novel solver~\cite{iter}. As well as this, Jacobi solvers cannot handle transient problems which result in systems of ODEs, so it would be interesting to apply some iterative ODE solvers also such as Runge-Kutta (or again a more novel solver) if plausible.
\subsubsection*{Summary of Approach:}
\begin{itemize}
\item Solve Finance-based PDE (as is personal background, actual PDE choice is trivial) using FEM in both serial and parallel.
\item Recreate FEM-SES in CUDA and benchmark against standard FEM done in previous step to demonstrates speedups/accuracy were as seen in the paper.
\item Apply and test more novel iterative linear system solvers to FEM-SES's Jacobi relaxation and benchmark accuracy and speed against standard FEM-SES.
\item Apply ODE iterative solver for a non-steady state/transient PDE and benchmark against traditional FEM methods.
\end{itemize}
\begin{thebibliography}{9}
\bibitem{femses} 
D.~Fernández, M.~M.~Dehnavi, W.~Gross, D.~Giannacopoulos, 
\textit{Alternate Parallel Processing Approach for FEM}, 
IEEE Transactions on Magnetics, (\textbf{48}), no.~2, 2012.
\bibitem{iter} 
J.~Wolfson-Pou, E.~Chow,
\textit{Distributed Southwell: An Iterative Method with Low
Communication Costs}, 
Proceedings of SC17, 2017.
\end{thebibliography}
\end{document}